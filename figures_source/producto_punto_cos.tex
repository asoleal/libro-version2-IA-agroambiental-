\documentclass[tikz, border=5pt]{standalone}
\usepackage[utf8]{inputenc}
\usepackage{tikz}
\usetikzlibrary{arrows.meta} % Necesario para >=Stealth

\begin{document}
\begin{tikzpicture}[scale=1.5, >=Stealth]
        % Definir coordenadas
        \coordinate (O) at (0,0);
        \coordinate (X) at (4,0);    % Vector x a lo largo del eje
        \coordinate (Y) at (2,2.5);  % Vector y formando un ángulo
        
        % Calcular la proyección de Y sobre X
        % Proyección escalar = |y| * cos(theta) = (x.y) / |x|
        % x=(4,0), y=(2,2.5). x.y = 8. |x|=4. Proyección = 8/4 = 2.
        \coordinate (P) at (2,0);    % Punto de proyección
        
        % --- Vectores Principales ---
        \draw[->, very thick, blue] (O) -- (X) node[right] {$\mathbf{x}$};
        \draw[->, very thick, red] (O) -- (Y) node[above] {$\mathbf{y}$};
        
        % --- Elementos de la Proyección ---
        % Línea punteada perpendicular
        \draw[dashed, gray, thick] (Y) -- (P);
        % Vector proyección (sombra)
        \draw[->, very thick, violet] (O) -- (P);
        
        % Marca de ángulo recto
        \draw[gray] (P) -- ++(0,0.2) -- ++(-0.2,0) -- ++(0,-0.2);
        
        % --- Ángulo theta ---
        \draw[thick, gray] (0.6,0) arc (0:51.34:0.6); % atan(2.5/2) approx 51.34 deg
        \node[gray] at (0.9, 0.4) {$\theta$};
        
        % --- Etiquetas de Magnitud ---
        % Etiqueta para la proyección
        \draw[<->, violet, thick] (0,-0.3) -- (2,-0.3) node[midway, below] {$\lVert \mathbf{y} \rVert \cos(\theta)$};
        % Etiqueta para la norma de x
        \draw[<->, blue, thick] (0,-0.8) -- (4,-0.8) node[midway, below] {$\lVert \mathbf{x} \rVert$};
        
        % --- Texto Explicativo ---
        \node[anchor=north west, align=left, draw, fill=gray!10, rounded corners] at (3.5, 3.5) {
            \textbf{Interpretación Geométrica:}\\
            El producto punto es la magnitud de un vector\\
            multiplicada por la proyección del otro sobre él.\\[0.5em]
            $\mathbf{x} \cdot \mathbf{y} = \underbrace{\lVert \mathbf{x} \rVert}_{\text{Longitud de } \mathbf{x}} \cdot \underbrace{(\lVert \mathbf{y} \rVert \cos \theta)}_{\text{Proyección de } \mathbf{y} \text{ en } \mathbf{x}}$
        };
        
    \end{tikzpicture}
\end{document}
